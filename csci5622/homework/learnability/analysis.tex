\documentclass[12pt]{article}
\usepackage{amsmath,listings,graphicx,caption,subfig,float,subfig}
\usepackage[margin=0.5in]{geometry}

\begin{document}
\noindent
Robert Werthman\\
CSCI 5622\\
Homework 5: Learnability\\

\section*{Problem 1}
As stated by the problem, we have $\epsilon = .15$--the error of the the hypothesis $h_i$.  The probability of hypothesis $h$ being outputted with error $\epsilon$ is the confidence of the hypothesis which is given by the equation $1 - \delta$.  In this particular problem, $1 - \delta = .95$.  This means $\delta = .05$.\\
\\
This problem has a finite, consistent hypothesis class $H$.  It is finite because each hypothesis $h_i$ is a triangle with 3 distinct vertices on the interval [0,99]--there is a maximum number of triangles.  It is consistent because each $h_i$ determines if a given a training example $x_i$ is inside or outside of its boundaries.  This is aligned with the concept $c$ of labeling each point positive or negative depending on whether that point is interior and exterior to the triangle boundary.  There is an algorithm $A$ that exists that given a point $x_i$ and 3 vertices that comprise a triangle, will tell if that point is inside or outside of the triangle boundary.\\
\\
Because this problem has a finite, consistent hypothesis class we can use the following equation to find the bound on training examples $m$ to ensure each hypothesis has a confidence of 95\% and error of .15.
\[
m \ge \frac{1}{\epsilon}(ln|H| + ln\frac{1}{\delta})
\]  
$|H|$ is the number of hypothesis which is equal to the number of traingles that can be found in the problem.  The number of triangles that can be found in a given interval is the total number of combinations of the total number of points in that interval made up of 3 vertices $\binom{n}{3}$.  $n$ in the case when the interval for both $x$ and $y$ [0,99] is 200 points.  This means there are $\binom{200}{3} = 1313400$ possible triangles.  This means $|H|$ = 1313400.  Plugging $|H|, \epsilon, \delta$ in to the equation for above we get
\[
m \ge \frac{1}{.15}(ln(1313400) + ln\frac{1}{.05}) \approx 114 \text{ training samples}
\]

\section*{Problem 2}
\begin{figure}[H]
\centering
  \subfloat[Lower VC Dim bound example]{
  \includegraphics[scale=.3]{p2_a.png}
  }
\centering
\end{figure}
The figure above is an example of a configuration of 2 points that are shattered by H.  Because this is the lowest number of points H can shatter, the $VCdim(H) \ge 2$. 

\end{document}