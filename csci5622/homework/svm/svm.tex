\documentclass[10pt]{article}
\usepackage{amsmath,listings,graphicx,caption,subfig,float,subfig}
\usepackage[margin=0.5in]{geometry}

\begin{document}
\noindent
Robert Werthman\\
CSCI 5622\\
Homework 3: SVM\\

\begin{enumerate}
  \item\textbf{Different values of C and two kernels}\\
  I used 5 different values of regularization parameter C: 0.1, 1, 10, 100,
  1000.  I used two different kernels: linear, radial basis function.  For each
  kernel I used all five values of C. I determined
  the accuracy for each value of C and kernel by using 3-fold cross validation
  with just the original training set.  The accuracy for each value of C and
  kernel is the mean of the accuracies for all of the cross validation folds.\\
  \\ 
  As can be seen in the figure below, the radial basis function kernel did
  better than the linear kernel for all values of C.  Increasing values of C,
  increased the accuracy of the radial basis function kernel while decreasing
  the accuracy of the linear kernel. 
  \begin{figure}[H]
    \centering
    \includegraphics[scale=.3]{performance_evaluation.png}
    \end{figure}
  \item \textbf{Examples of support vectors for each class}\\
  Example of three support vectors for the 3's class:
    \begin{figure}[H]
    \centering
      \subfloat{
      \includegraphics[scale=.2]{3_sv0.png}
      }
      \subfloat{
      \includegraphics[scale=.2]{3_sv1.png}
      }
      \subfloat{
      \includegraphics[scale=.2]{3_sv2.png}
      }
     \caption{3's support vectors}
    \end{figure}
    Example of three support vectors for the 8's class:
    \begin{figure}[H]
    \centering
      \subfloat{
      \includegraphics[scale=.2]{8_sv0.png}
      }
      \subfloat{
      \includegraphics[scale=.2]{8_sv1.png}
      }
      \subfloat{
      \includegraphics[scale=.2]{8_sv2.png}
      }
     \caption{8's support vectors}
    \end{figure}
\end{enumerate}

\end{document}